% Allgemeine Einleitung und Erklärung
\section{Allgemeine Projekt Beschreibung}
In der folgenden Konzeptionierung wird eine Umrichteranlage an $\SI{110}{\kV}$, im $\SI{50}{\Hz}$ Drehstrom Netz für das $\SI{110}{\kV}$,$\SI{16.7}{\Hz}$ Bahnnetz ausgelegt.
Die Einspeisung aus dem Drehstromnetz erfolgt über einen Netztrafo, dessen sekundäre Wicklungen jeweils mit \gls{4QS} verknüpft sind. 

Der Zwischenkreis verfügt über einen Widerstandssteller, Zwischenkreiskondensatoren und einem $\SI{33.3}{\Hz}$ Saugkreisfilter. 
Die Einspeisung ins $\SI{110}{\kV}$ Bahnnetz erfolgt über einen Bahntransformator mit jeweils vier Wicklungen auf der Primär- und Sekundärseite. 
Der Bahntransformator wird vom Zwischenkreis über jeweils einen \gls{4QS} pro Wicklung gespeist. 

Für den Zwischenkreis ist zusätzlich ein Vorladungs- und Erdungssystem vorgesehen, das aus einem Gleichrichter, der aus dem $\SI{400}{\V}$ Drehstromnetz gespeist wird, und einem Leistungsschalter gegen Erde besteht.
Ein Übersichtsschaltplan der Anlage ist im Anhang hinterlegt.
