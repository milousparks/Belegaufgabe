\section{Konzeptvergleich: Bahnumrichteranlagen}

Für eine Umrichteranlage zur Versorgung des Bahnstromnetzes aus dem Drehstromnetzt, können verschiedenen Konzepte zum Einsatz kommen.
Im Folgendem sollen diese Konzepte aus technischer und komerzieller Sicht Verglichen werden.

Es soll hier auf zwei Prinzipien eingegangen werden:

\textcolor{blue}{\textbf{Rotierender Umformer:}}
\\
Bei rotierenden Umformern werden in der Regel auf der Drehstromseite eine Dreiphasen-Asycnchronmaschiene mit der dreifachen Polzahl gegenüber der Einphasen-Synchronmaschiene auf der Bahnetzseite verwendet.

\textcolor{blue}{\textbf{Stationäre Umrichter:}}

Bei stationären Umrichtern kommt Halbleitertechnik zum Einsatz, um die benötigten Spannungen zu erzeugen. Bei indirekten Umrichtern wird, bei einem Energifluss ins Bahnnetz, mit einer Gleichrichter-Zwischenkreis-Wechselrichter
Topologie gearbeitet. 

\textcolor{blue}{\textbf{Vergleich der Konzepte:}}

\begin{tabular}{ l|l  }
   
    Rotierender Umformer & Stationäre Umrichter \\
    \hline
    \textbullet  \textcolor{red}{Komplexes bauliches Projekt (rotierende Massen)} & \textbullet  \textcolor{ForestGreen}{Einfacher Aufbau z.B. Container}\\
    \textbullet  \textcolor{red}{Hoher Wartungsaufwand (bemannt)} & \textbullet \textcolor{ForestGreen}{Geringer Wartungsaufwand} \\
    \textbullet \textcolor{ForestGreen}{Verfügbarkeit $\approx 93\%$} & \textbullet \textcolor{ForestGreen}{Verfügbarkeit $\approx 98\%$} \\ 
    \textbullet \textcolor{red}{Wirkungsgrad $\approx 92\%..95\%$} & \textbullet \textcolor{ForestGreen}{ Wirkungsgrad $\approx 97.5\%$  }\\
    \textbullet \textcolor{red}{ Dynamik begrenzt (rotierende Massen) $\approx \SI{10}{\MW\per\second}$}& \textbullet \textcolor{ForestGreen} {Hohe Dynamik $<\SI{500}{\MW\per\second}$} \\
    \textbullet \textcolor{ForestGreen}{Überlastbar (Netzstabilisierend)} & \textbullet \textcolor{red}{Geringe Überlastbarkeit }\\
    \textbullet \textcolor{ForestGreen}{4-Facher Kurschlussstrom }& \textbullet \textcolor{red}{$1.3$-Facher Kurzschlussstrom }\\
\end{tabular}


Der Stationäre Umrichter biete gegenüber dem rotierenden Umformer viele technische sowie monetäre Vorteile. 
Besonders der wartungsarm Betrieb und der bessere Wirkungsgrad wirken sich auf die laufenden Kosten aus.
Bei einem unterschied von $\Delta\eta\approx 5\%$ und einer Nennleistung von $P=\SI[]{17.5}{MW}$ hat der rotierende Umformer 
eine zusätzlichen Verlust von $\Delta P=\SI[]{875}{\kilo\watt}$. 
In einem Jahr Betrieb fallen damit $W=\SI{7.665}{\giga\watt\hour}$ zusätzliche Verlsutleistung an.

