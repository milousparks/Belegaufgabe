%Dokumenteinstellungen und Anpassungen
%Dokumentenklasse "scrbook" - Erweitert um den Verweis auf die Verzeichnisse und Texteigenschaften
\documentclass[12pt,a4paper,oneside,parskip=half,listof=totoc,bibliography=totoc,numbers=noendperiod,ngerman,dvipsnames]{scrartcl}
%Anpassung der Seitenränder (Standard bottom ca. 52mm anbzüglich von ca. 4mm für die nach oben rechts gewanderte Seitenzahl)
\usepackage[bottom=48mm,left=25mm,right=25mm]{geometry}

%Tweaks für scrbook
\usepackage{scrhack}

%Blindtext
%\usepackage{blindtext}

%Erlaubt unteranderem Umbrücke captions
\usepackage{caption}
\usepackage{derivative}
%Stichwortverzeichnis
\usepackage{imakeidx}
\usepackage{amsmath}
%Kompakte Listen
\usepackage{paralist}
% Multiline comments
\usepackage{comment}
%Zitate besser formatieren und darstellen
\usepackage{epigraph}
\usepackage{acronym}

%Glossar, Stichworverzeichnis (Akronyme werden als eigene Liste aufgeführt)
\usepackage[toc, acronym]{glossaries} 

%Anpassung von Kopf- und Fußzeile
%beinflusst die erste Seite des Kapitels
\usepackage[automark,headsepline]{scrlayer-scrpage}
%\automark{chapter}
\ihead{\leftmark}
\chead{}
\ohead{\thepage}
\ifoot*{Milan Daniel Larsen 581929}
\cfoot[\thepage]{}
\cfoot*{}
\ofoot*{\thepage}
\ofoot{}
\cfoot{}
\pagestyle{scrheadings}


%Auskommentieren für die Verkleinerung des vertikalen Abstandes eines neuen Kapitels
%\renewcommand*{\chapterheadstartvskip}{\vspace*{.25\baselineskip}}

%Zeilenabstand 1,5
\usepackage[onehalfspacing]{setspace}

%Verbesserte Darstellung der Buchstaben zueinander
\usepackage[stretch=10]{microtype}

%Deutsche Bezeichnungen für angezeigte Namen (z.B. Innhaltsverzeichnis etc.)
\usepackage[ngerman]{babel}

%Unterstützung von Umlauten und anderen Sonderzeichen (UTF-8)
\usepackage{lmodern}
\usepackage[utf8]{luainputenc}
\usepackage[T1]{fontenc}

%Einfachere Zitate
\usepackage{epigraph}

%Unterstützung der H positionierung (keine automatische Verschiebung eingefügter Elemente)
\usepackage{float} 

%Erlaubt Umbrüche innerhalb von Tabellen
\usepackage{tabularx}

%Erlaubt Seitenumbrüche innerhalb von Tabellen
\usepackage{longtable}

%Erlaubt die Darstellung von Sourcecode mit Highlighting
\usepackage{listings}

%Definierung eigener Farben bei nutzung eines selbst vergebene Namens
\usepackage[table,xcdraw]{xcolor}
\usepackage{pdfpages}
%Vektorgrafiken
\usepackage{tikz}

%Grafiken (wie jpg, png, etc.)
\usepackage{graphicx}

%Grafiken von Text umlaufen lassen
\usepackage{wrapfig}
\usepackage{pgfplots}

%Ermöglicht Verknüpfungen innerhalb des Dokumentes (e.g. for PDF), Links werden durch "hidelink" nicht explizit hervorgehoben
\usepackage[hidelinks,german]{hyperref}
\usepackage{cleveref}
\usepackage{siunitx}
\usepackage{physics}
\usepackage[most]{tcolorbox}
\newtcolorbox{mybox}{
	enhanced,
	boxrule=0pt,frame hidden,
	borderline west={4pt}{0pt}{green!75!black},
	colback=green!10!white,
	sharp corners
}
\newtcolorbox{qed}{
	enhanced,
	boxrule=0pt,frame hidden,
	borderline west={4pt}{0pt}{blue!75!black},
	colback=cyan!5!white,
	sharp corners
}
\usepackage{todonotes}
\usepackage{cancel}%Durchstreichen von sachen
\usepackage[numbered,framed]{matlab-prettifier}
%Einbindung und Verwaltung von Literaturverzeichnissen
\usepackage{csquotes} %wird von biber benötigt
\usepackage[style=ieee, backend=biber, bibencoding=ascii]{biblatex}
\addbibresource{references/references.bib}
\usepackage{multirow}
\usepackage{nicematrix}

\newenvironment{conditions}[1][mit:]
{#1 \begin{tabular}[t]{>{$}l<{$} @{${}={}$} l}}
	{\end{tabular}\\[\belowdisplayskip]}

\usepackage{fancyvrb}
\lstset{literate=% Allow for German characters in lstlistings.
	{Ö}{{\"O}}1
	{Ä}{{\"A}}1
	{Ü}{{\"U}}1
	{ß}{{\ss}}2
	{ü}{{\"u}}1
	{ä}{{\"a}}1
	{ö}{{\"o}}1
}
\usepackage{booktabs}
\usepackage{caption}
\usepackage{subcaption}

\definecolor{gray}{RGB}{127,127,127}
\definecolor{lightgray}{RGB}{219,219,219}
\definecolor{lightergray}{RGB}{237,237,237}

\DeclareSIUnit \voltampere {VA} %apparent power 
\DeclareSIUnit \VA {VA} %apparent power 
\DeclareSIUnit \var { var } %volt-ampere reactive - idle power 
\input{resources/styles/adjustments}

%Titelformen - gewünschtes Layout einkommentieren

%%Graduation
\makeatletter

\newcommand*{\gradeType}[1]{\gdef\@gradeType{#1}}
\newcommand*{\firstExaminer}[1]{\gdef\@firstExaminer{#1}}
\newcommand*{\secondExaminer}[1]{\gdef\@secondExaminer{#1}}
\newcommand*{\matrikelnr}[1]{\gdef\@matrikelnr{#1}}
\newcommand*{\submitDate}[1]{\gdef\@submitDate{#1}}

\renewcommand*{\maketitle}{
	\begin{titlepage}
	\newgeometry{left=2.5cm,right=2.5cm,top=3.0cm,bottom=2.5cm}
	
	\begin{minipage}{0.5\textwidth}
		\includegraphics[height=0.1\textheight]{./Bilder/FB1_ET_rgb}
	\end{minipage}
	\begin{minipage}{0.5\textwidth}
		\begin{flushright}
			\includegraphics[height=0.1\textheight]{./Bilder/HTW_Logo_rgb}
		\end{flushright}
		
	\end{minipage}

		\begin{center}
			\vfill
			{\Large \@title\par}
			\vskip 0.5cm
			{\large \bfseries Hausarbeit\par}
			\vskip 0.5cm
			{\large im Studienfach\\ \bfseries \@gradeType}
			\vskip 0.5cm
			{\large an der}
			\vskip 0.5cm
			{\large Hochschule für Technik und Wirtschaft Berlin\\ Fachbereich  I Energie und Information\\ Studiengang Elektotechnik }
			\vfill
			\begin{flushleft}
				\begin{tabular}[t]{rl}
					1. Prüfer: &\@firstExaminer\\
					\\					
					Eingereicht von: &Reebal Nofal\\
					Matrikelnummer:	&563040\\
					Eingereicht von: &\@author\\
					Matrikelnummer: & \@matrikelnr\\
									
					Datum der Abgabe: & \@submitDate
				\end{tabular}
			\end{flushleft}
		\end{center}
		\restoregeometry
	\end{titlepage}
}
\makeatother
\gradeType{Angewandte Mathematik}
\secondExaminer{Max Mustermann}

%Research paper
%\include{titles/research_papger}
%\subTitle{Ein optionaler Untertitel der Arbeit}
%\researchPart{A}

%Angaben zur Arbeit und dem Author (von beiden Layouts genutzt)
\title{Projekt Zeitaufgelöste Photolumineszenz}
\author{Milan Daniel Larsen}
\matrikelnr{581929}
\submitDate{30.1.2022}
\firstExaminer{Prof. Dr. Andreas Zeiser}

%Verzeichnisse generieren
\makeglossaries
\loadglsentries{references/glossary_acronyms.tex}
\setacronymstyle{long-short}

\makeindex[columns=2, title=Stichwortverzeichnis, options= -s resources/styles/indexstyle.ist, intoc]
\indexsetup{level=\chapter*,toclevel=chapter}

%Start des Inhalts
\begin{document}

%Notwendiger Workaround
\pagenumbering{alph}

%Deckblatt erzeugen
\maketitle

\pagenumbering{Roman}

%\include{chapter/Vorwort} \clearpage
%\include{chapter/Abstract} \clearpage

%Inhaltsverzeichnis
\tableofcontents \newpage

%Hauptteil
\pagenumbering{arabic}
% Allgemeine Einleitung und Erklärung
\section{Anlagenkonzept}
\subsection{Vergleich: TCR vs Statcom }
\section{Stromlaufplan}
\subsection{ Single-Line-Diagramm: Dynamischer Kompensationsanlage}
\section{Auslegung}
\subsection{Berechnungen}
\subsection{Filterkreiskondensator}  
\subsection{Filterkreisdrossel}
\subsection{Thyristorgesteuerte Drossel}
\subsection{Leistungsschalter}
\section{Detailbeschreibung}
\subsection{Erläuterungsbericht Auslegungsberechnungen}
\subsection{TCR-Regelstrategie}

\section{Literatur}
\subsection{SIEMENS 3AH3 Katalog} \clearpage
\section{Konzeptvergleich: Bahnumrichteranlagen}

Für eine Umrichteranlage zur Versorgung des Bahnstromnetzes aus dem Drehstromnetzt, können verschiedenen Konzepte zum Einsatz kommen.
Im Folgendem sollen diese Konzepte aus technischer und komerzieller Sicht Verglichen werden.

Es soll hier auf zwei Prinzipien eingegangen werden:

\textcolor{blue}{\textbf{Rotierender Umformer:}}
\\
Bei rotierenden Umformern werden in der Regel auf der Drehstromseite eine Dreiphasen-Asycnchronmaschiene mit der dreifachen Polzahl gegenüber der Einphasen-Synchronmaschiene auf der Bahnetzseite verwendet.

\textcolor{blue}{\textbf{Stationäre Umrichter:}}

Bei stationären Umrichtern kommt Halbleitertechnik zum Einsatz, um die benötigten Spannungen zu erzeugen. Bei indirekten Umrichtern wird, bei einem Energifluss ins Bahnnetz, mit einer Gleichrichter-Zwischenkreis-Wechselrichter
Topologie gearbeitet. 

\textcolor{blue}{\textbf{Vergleich der Konzepte:}}

\begin{tabular}{ l|l  }
   
    Rotierender Umformer & Stationäre Umrichter \\
    \hline
    \textbullet  \textcolor{red}{Komplexes bauliches Projekt (rotierende Massen)} & \textbullet  \textcolor{ForestGreen}{Einfacher Aufbau z.B. Container}\\
    \textbullet  \textcolor{red}{Hoher Wartungsaufwand (bemannt)} & \textbullet \textcolor{ForestGreen}{Geringer Wartungsaufwand} \\
    \textbullet \textcolor{ForestGreen}{Verfügbarkeit $\approx 93\%$} & \textbullet \textcolor{ForestGreen}{Verfügbarkeit $\approx 98\%$} \\ 
    \textbullet \textcolor{red}{Wirkungsgrad $\approx 92\%..95\%$} & \textbullet \textcolor{ForestGreen}{ Wirkungsgrad $\approx 97.5\%$  }\\
    \textbullet \textcolor{red}{ Dynamik begrenzt (rotierende Massen) $\approx \SI{10}{\MW\per\second}$}& \textbullet \textcolor{ForestGreen} {Hohe Dynamik $<\SI{500}{\MW\per\second}$} \\
    \textbullet \textcolor{ForestGreen}{Überlastbar (Netzstabilisierend)} & \textbullet \textcolor{red}{Geringe Überlastbarkeit }\\
    \textbullet \textcolor{ForestGreen}{4-Facher Kurschlussstrom }& \textbullet \textcolor{red}{$1.3$-Facher Kurzschlussstrom }\\
\end{tabular}


Der Stationäre Umrichter biete gegenüber dem rotierenden Umformer viele technische sowie monetäre Vorteile. 
Besonders der wartungsarm Betrieb und der bessere Wirkungsgrad wirken sich auf die laufenden Kosten aus.
Bei einem unterschied von $\Delta\eta\approx 5\%$ und einer Nennleistung von $P=\SI[]{17.5}{MW}$ hat der rotierende Umformer 
eine zusätzlichen Verlust von $\Delta P=\SI[]{875}{\kilo\watt}$. 
In einem Jahr Betrieb fallen damit $W=\SI{7.665}{\giga\watt\hour}$ zusätzliche Verlsutleistung an.

 \clearpage
\section{Drehstrom-Leistungstransformator 50 Hz}
Der Transformator soll für die Außenaufstellung ausgelegt werden und wird von 3 AC \SI[]{50}[]{\Hz}, \SI[]{110}[]{\kilo\volt} gespeist.
Der Transformator soll ölgefüllt und selbstkühlend sein.\\ 

\textbf{Schaltbild}
\begin{figure}[htb]
\centering
\includegraphics[width=\textwidth/2,frame]{Bilder/netztrafo.png}
\end{figure}

\subsection{Allgemeine Merkmale}

\begin{table}[htb]
    \centering
    \begin{NiceTabular}{|l|c|}[]
        \CodeBefore
        \columncolor{lightergray}{1}
        \Body
        \hline
         Aufstellung & Freiluftaufstellung\\
         \hline
         Verschmutzung & Verschmutzungsgrad III (stark) \\
         \hline
         Aufstellungshöhe & < 1000 m üNN\\
         \hline
         Umgebungstemperatur &  -30°C bis 40°C\\
         \hline
         Klimabedingungen & Normal\\ 
         \hline
                 \Block{3-1}{Dokumentationen} &  \tabitem Technische Zeichnungen und CAD\\
                         &\tabitem Montageplan, Wartungsplan, Dokumentationen\\
                         &\tabitem Prüfprotokoll der zu erfüllenden Prüfungen\\
                         \hline
       
    \end{NiceTabular}
\end{table}

\pagebreak
\subsubsection*{Normen}

\begin{itemize}[noitemsep]
    \item DIN VDE 0532-76-1: Leistungstransformatoren
    \item DIN EN 61378-1 Stromrichtertransformatoren - Teil 1: Transformatoren für industrielle Anwendungen
    \item DIN EN 60076-3 Leistungstransformatoren Teil 3: Isolationspegel, Spannungsprüfungen und äußere Abstände in Luft
\end{itemize}


\subsection{Bemessungsdaten:}

\begin{table}[htb]
    \centering
    \begin{NiceTabular}{|l|p{2cm}|p{2cm}|}[hvlines]
        \CodeBefore
        \columncolor{lightergray}{1}
        \Body
       \Block{2-1}{Schaltgruppe} & OS & US \\ 
                                & Y(N) &   i0i0i0  \\
         Nennleistung ohne Leistung der Filterwicklung & \Block{1-2}{$\SI{17.68}{\unit{\mega\volt\ampere}}$}\\
         Nennspannung OS (Klemmenspannung) & \Block{1-2}{$\SI{110}{\kV}$}\\
         Max. Spannung OS (Klemmenspannung) & \Block{1-2}{$\SI{123}{\kV}$}\\
         Nennspannung US (Klemmenspannung) & \Block{1-2}{$\SI{3536}{\V}$}\\
         Nennstrom der US bei Nennspannung & \Block{1-2}{$\SI{1.667}{\kilo\ampere}$}\\
    \end{NiceTabular}
\end{table}

\textbf{Relative Kurzschlussspannungen:}
\begin{itemize}
    \item Bezugsgrößen: \\ bezogen auf Nennleistung bei $\SI{75}{\degree}$C; eine US Wicklung kurzgeschlossen; alle anderen Wicklungen offen; Speisung in OS Wicklung
    \item Werte \\ $uk_\mathrm{OS_iUS_i}\mathrm{(mit\,i=1...3)}=20\% (20.9\%...23.1\%)$; bezogen auf Nennleistung\\ $uk_\mathrm{US-US}>22\%$ (für alle Paarungen)
\end{itemize}

\subsubsection*{Verluste}
\begin{table}[htb]
    \centering
    \begin{NiceTabular}{|l|c|c|}[hvlines]
        \CodeBefore
        \columncolor{lightergray}{1}
        \Body
        &Grundschwingung&Umrichterbetrieb (Zusatzverluste)\\
        Leerlaufverluste bei Nennspannung &  tbd. kW&<1\% von der Grundschwingung\\
        Kurzschlußverluste bei 75 °C &tbd. kW&<1\% von der Grundschwingung\\
    \end{NiceTabular}
\end{table}
\clearpage

\subsubsection*{Stromwandler}

\begin{table}[!htb]
    \centering
    \begin{NiceTabular}{|l|c|}[hvlines]
        \CodeBefore
        \columncolor{lightergray}{1}
        \Body
        \Block{2-1}{Stromwandler OS-Seite} &  3xtbd/1A; 15VA; 10P10\\
                                & 3xtbd/1A; 15VA; 0,5 FS10\\
                                Stromwandler US-Seite &3x tbd/1A;15VA;10P10\\
                                Stromwandler Kesselschutz &1x 100/1A;3VA;5P20\\
    \end{NiceTabular}
\end{table}

\subsubsection*{Durchführungen}
\begin{table}[h]
    \centering
    \begin{NiceTabular}{|l|c|}[hvlines]
        \CodeBefore
        \columncolor{lightergray}{1}
        \Body
        OS& 3 (+1 optional Sternpunkt herausführbar)\\
        US & 3x2 \\
    \end{NiceTabular}
\end{table}

\textbf{Isolation (nach Prüfungsnorm in \cite*{DINEN600763VDE0532763:201903.}):}
\begin{table} [h]
    \centering
    \begin{NiceTabular}{|l|c|c|}[hvlines]
        \CodeBefore 
        \columncolor{lightergray}{1}
        \Body
             & OS & US gegen Erde \\ 
           max. Betriebsspannung  & $\SI{123}{\kilo\volt}$ &  $\SI{7.2}{\kilo\volt}$ \\
         Nennstehwechselspannung & $U_1=$\SI{185}{\kilo\volt}; $U_2=$\SI{230}{\kilo\volt}& \SI{20}{\kilo\volt} \\
         Nennstehblitzspannung & $U_1=$\SI{450}{\kilo\volt}; $U_2=$\SI{550}{\kilo\volt}&$U_1=$\SI{40}{\kilo\volt}; $U_2=$\SI{60}{\kilo\volt}\\
    \end{NiceTabular}
\end{table}

\subsubsection*{Sternpunktausführung}
Der Sternpunkt OS ist aus der Wicklung herauszuführen und eine spätere Verwendung vorzubereiten. Durchführung und Isolator sind nicht erforderlich, der Sternpunkt kann blind verflanscht werden. 

\textbf{Kapazitive Kopplung}\\
Eine kapazitive Übertragung von Blitzüberspannungen von der OS-Wicklung auf die US-Wicklung ist zu vermeiden. Bisherige Transformatoren in Bahnkupplungen hatten zu diesem Zweck Schirmwicklungen. 
 
\textbf{Geräuschpegel}\\
Aufstellungsort: Allgemeines Wohngebiet gemäß § 1 BImSchG  $L_\mathrm{pmax}=40 dB(A)$.
Grenzwert darf im Fernfeld(5m) mit Messung nach DIN EN 60076-10 nicht überschritten werden.
 \clearpage
\section{Einphasen-Stromrichteröltrafo 16.7 Hz}
Der \SI[]{16.7}[]{\Hz} Transformator ist ein Summiertransformator und addiert die
Teilspannungen der Umrichterr auf die Bahnspannung 2 AC
\SI[]{110}[]{\kV}. Der Transformator ist ölgefüllt, selbstkühlend und für die Aussenaufstellung ausgelegt.

\subsection{Allgemeine Merkmale}

\begin{table}[htb]
    \centering
    \begin{NiceTabular}{|l|c|}[]
        \CodeBefore
        \columncolor{lightergray}{1}
        \Body
        \hline
         Aufstellung & Freiluftaufstellung\\
         \hline
         Verschmutzung & Verschmutzungsgrad III (stark) \\
         \hline
         Aufstellungshöhe & < 1000 m üNN\\
         \hline
         Umgebungstemperatur &  -30°C bis 40°C\\
         \hline
         Klimabedingungen & Normal\\ 
         \hline
                 \Block{3-1}{Dokumentationen} &  \tabitem Technische Zeichnungen und CAD\\
                         &\tabitem Montageplan, Wartungsplan, Dokumentationen\\
                         &\tabitem Prüfprotokoll der zu erfüllenden Prüfungen\\
            \hline
    \end{NiceTabular}
\end{table}

\textbf{Normen:}

\textbf{Schaltbild}
\begin{figure}[htb]
\centering
\includegraphics[width=\textwidth/3,frame]{Bilder/stromrichtertrafo.png}
\end{figure}

\textbf{Bemessungsdaten:}
\begin{table}[htb]
    \centering
    \begin{NiceTabular}{|l|p{2cm}|p{2cm}|}[hvlines]
        \CodeBefore
        \columncolor{lightergray}{1}
        \Body
       \Block{2-1}{Schaltgruppe} & OS & US \\ 
                                & | &   i0i0i0  \\
         Nennleistung ohne Leistung der Filterwicklung & \Block{1-2}{$\SI{20}{\unit{\mega\volt\ampere}}$}\\
         Leistung US Wicklung & \Block{1-2}{$4\cdot\SI{5.125}{\mega\VA}$}\\
         Nennfrequenz nach DIN EN 50163/A1 \cite{DeutschesInstitutfurNormungene.V..200802} & \Block{1-2}{$\SI{16.7}{\Hz}-6\%+4\%$}\\
         Nennspannung der OS-Wicklung  & \Block{1-2}{$\SI{110}{\kilo\V}$}\\
         Nennspannung einer US Wicklung bei \SI[]{110}[]{\kV} & \Block{1-2}{$4\cdot\SI{3535}{\kV}$}\\
         Nennstrom US-Wicklung bei Nennspannung & \Block{1-2}{$\SI{1414}{\A}$}\\
        Filterwicklung (HW) Nennleistung & \Block{1-2}{$\SI{4.8}{\mega\VA}$}\\
        Filterwicklung (HW) Nennspannung & \Block{1-2}{$\SI{6}{\kilo\V}$}\\
    \end{NiceTabular}
\end{table}

\textbf{Kurzschlussspannung, Impedanzen}
\begin{table}[htb]
    \centering
    \begin{NiceTabular}{|l|p{2cm}|p{2cm}|}[hvlines]
        \CodeBefore
        \columncolor{lightergray}{1}
        \Body
       \Block{2-1}{Schaltgruppe} & OS & US \\ 
                                & Y(N) &   i0i0i0  \\
         Nennleistung ohne Leistung der Filterwicklung & \Block{1-2}{$\SI{20}{\unit{\mega\volt\ampere}}$}\\
         Leistung US Wicklung & \Block{1-2}{$4\cdot\SI{5.125}{\mega\VA}$}\\
         Nennfrequenz nach DIN EN 50163/A1 \cite{DeutschesInstitutfurNormungene.V..200802} & \Block{1-2}{$\SI{16.7}{\Hz}-6\%+4\%$}\\
         Nennspannung der OS-Wicklung  & \Block{1-2}{$\SI{110}{\kilo\V}$}\\
         Nennspannung einer US Wicklung bei \SI[]{110}[]{\kV} & \Block{1-2}{$4\cdot\SI{3535}{\kV}$}\\
         Nennstrom US-Wicklung bei Nennspannung & \Block{1-2}{$\SI{1414}{\A}$}\\
        Filterwicklung (HW) Nennleistung & \Block{1-2}{$\SI{4.8}{\mega\VA}$}\\
        Filterwicklung (HW) Nennspannung & \Block{1-2}{$\SI{6}{\kilo\V}$}\\
    \end{NiceTabular}
\end{table}
 \clearpage
\section{Berechnungen}

\textbf{Nennleistung ohne Leistung der Filterwicklung:}
\begin{equation}
    S_\mathrm{N}=\frac{P_\mathrm{N}}{cos \Phi_\mathrm{max}}=\frac{\SI{16}{\MW}}{0.8}=\SI{20}{\mega\voltampere} 
\end{equation} \clearpage

\newacronym{4QS}{4QS}{Vierquadrantensteller}
%Anhang
\pagenumbering{Alph}

%Abbildungsverzeichnis
%\listoffigures \clearpage
%Tabellenverzeichnis
%\listoftables \clearpage
%Quelltextverzeichnis
%\lstlistoflistings \clearpage
%Stichwortverzeichnis
%\printindex \clearpage
%Glossar
%\printglossary[title={Glossar}] \clearpage
%Abkürzungsverzeichnis
%\printglossary[style=dottedlocations,type=\acronymtype,title={Abkürzungsverzeichnis}] \clearpage

%Literaturverzeichnisse (getrennt nach Stichwort)
%\printbibliography[heading=bibintoc, keyword={book}, %title={Literaturverzeichnis}]\clearpage
%\printbibliography[heading=bibintoc, keyword={online}, title={Onlinequellen}]\clearpage
%\printbibliography[heading=bibintoc, keyword={image}, title={Bildquellen}]\clearpage
\printbibliography[title={Normen}]\clearpage
% Anhang
\appendix
\section{Anhang}
\subsection{Anforderungen der DB AG}
Anforderungen an die Transformatoren aus der Technischen Spezifikation der Frequenzumrichter
der DB AG vom 07.05.2009: 
"5.9 Transformatoren und Trafowannen“

\subsubsection*{Korrosionsschutz (Anstrich)}
Zur Gewährleistung der Mindestlebenserwartung von 30 Jahren ist eine Oberflächenbehandlung von Kessel, Ausdehnungsgefäß, Radiatoren und Rohrleitungen notwendig. Die Anstriche sind nach TL 918 300, Bl. 87 auszuführen. Korrosionsbeständige Bauteile sind ebenfalls in die Oberflächenbehandlung mit einzubeziehen (Grund- / Haftbeschichtung und die beiden Deckanstriche).
Feuerverzinkte Bauteile sind besonders sorgfältig zu behandeln.
Die Verwendung von alternativen Farbherstellern oder das Aufbringen von Wasserlacken ist mit der 
DB Energie abzustimmen.

\subsubsection*{Oberflächenbehandlung von Kessel, Ausdehnungsgefäß, Radiatoren und Rohrleitungen}
Die Komponenten sind mit einer Spritzverzinkung in der Mindestschichtendicke von \SI{100}{\um}, nach DIN-VDE 55 928 Teil 5 oder mit einer Epoxyd-Zinkstaubfarbe, DB Mat.-Nr. 687.03, mit einer abschließenden Mindesttrockenschichtdicke von \SI[]{80}{\um}, zu behandeln. Nach dem metallischen Überzug ist eine Grund- / Haftbeschichtung aufzutragen.

\subsubsection*{Erste Deckbeschichtung}
Die erste Deckbeschichtung ist mit Epoxyd-Eisenglimmer Anstrich in der Farbe grau, Farbton DB 702, Mat.-Nr. 687.12, auszuführen. Die Mindesttrockenschichtdicke beträgt \SI{80}{\um}.

\subsubsection*{Zweite Deckbeschichtung}
Die zweite Deckbeschichtung ist mit PUR-Eisenglimmer Anstrich in der Farbe grün, Farbton DB 601, Mat.-Nr. 687.61, auszuführen. Die Mindesttrockenschichtdicke beträgt \SI{80}{\um}.

\subsubsection*{Dichtigkeit}
Für die Öldichtigkeit beträgt die Verjährungsfrist für Sachmängel 5 Jahre nach Inbetriebnahme.
Desgleichen beträgt die Verjährungsfrist für Sachmängel 5 Jahre für den aufgebrachten Korrosionsschutz

\subsubsection*{Beschilderung}
Die Beschilderung der Armaturen hat entsprechend DIN 42513 “Bauteilkennzeichnung für Transformatoren und Drosselspulen“ zu erfolgen. Es sind alterungsbeständige Kennzeichnungsschilder zu verwenden.

\subsubsection*{Alterungsbeständigkeit}
Alle eingesetzten Kunststoffe müssen UV-, alterungs- und wetterbeständig ausgeführt sein. Dies gilt insbesondere für eingesetzte Kabel, Kabelbinder und angebrachte Kennzeichnungsschilder.

\subsubsection*{Überspannungsschutz}
Für den Überspannungsschutz der Umrichtertrafos und -komponenten sind auf der 50-Hz- und 16,7-Hz-Seite Metalloxid-Überspannungsableiter vorzusehen. Sollten diese aus Sicht des Herstellers nicht notwendig sein, ist dies schriftlich gegenüber dem Auftraggeber zu begründen (Siehe auch 7.3).

\subsubsection*{Absturzsicherung}
Für alle Bauteile (Trafos, Container, etc.) ist eine Absturzsicherung mitzuliefern. Die Bestimmungen des Arbeitsschutzes sind einzuhalten.

\subsubsection*{Luftabschluss}
Alle Transformatoren sind mit Luftabschlusssystemen (Luftsack, Ausdehnung über Radiatoren) zu versehen.

\subsubsection*{Transformator - Ölauffangwanne}
Ausführung gemäß den gelten Gesetzen (z. B. Wasserhaushaltsgesetz) als Fertigwannen. Transformatorenfundament aus werksmäßig hergestellten Fertigelementen aus Stahlbeton gem. DIN 1045 C35/45 FD mit den Expositionsklassen XC4, XF3, XA2, Dichtigkeitsnachweis gem. DafStb-Richtlinie “Betonbau beim Umgang mit Wassergefährdenden Stoffen“, monolithisch hergestellt, Boden und Umfassungswände fugenlos aus einem Guss. Die Auffangwannen erfüllen die Anforderungen des Wasserhaushaltsgesetzes (WHG) §19."
\subsection{Harmonische von Umrichterspannungen und Umrichterströmen}
\subsubsection{Oberschwingungsamplituden der Drehstromseite}
Unten werden die maximalen Oberschwingungsamplituden bk der Spannung angegeben, die vom Umrichter an den umrichterseitigen Wicklungssystemen des Drehstromtransformators eingeprägt werden. Der zugehörige Zeitverlauf ergibt sich aus:

\begin{equation*}
    u_\mathrm{k}(t)=b_\mathrm{k}sin(k\cdot \Omega t)
\end{equation*}

\begin{figure}[htb]
    \centering
    \includegraphics[width=\textwidth]{Bilder/spannungsoberschwingung_umrichter.png}
    \caption{Maximale Oberschwingungsamplituden der Umrichterspannung der Drehstromseite über den Aussteuerungsbereich von 0,9 … 1,1 (100\% entsprechen 3062V verkettet effektiv)}
\end{figure}

\begin{table}[!ht]
    \centering
    \caption[]{Der Gesamt-THD beträgt etwa \SI{29}{\percent} je Wicklung. Die Spannungsamplituden sind zusätzlich als verkettete Effektivwerte angegeben.}
    \begin{tabular}{|l|l|l|}
    \hline
        Ordnungszahl & Oberschwingungsamplitude [\%] & Oberschwingungsamplitude [V] \\ \hline
        5 & 6.98 & 213.72 \\ \hline
        7 & 11.30 & 345.99 \\ \hline
        11 & 21.38 & 654.63 \\ \hline
        13 & 6.15 & 188.30 \\ \hline
        17 & 9.22 & 282.30 \\ \hline
        19 & 13.40 & 410.29 \\ \hline
        23 & 8.28 & 253.52 \\ \hline
        25 & 3.39 & 103.80 \\ \hline
        29 & 7.27 & 222.60 \\ \hline
        31 & 8.01 & 245.26 \\ \hline
        35 & 5.37 & 164.42 \\ \hline
        37 & 6.48 & 198.41 \\ \hline
        41 & 4.33 & 132.58 \\ \hline
        43 & 5.06 & 154.93 \\ \hline
        47 & 4.50 & 137.78 \\ \hline
        49 & 4.64 & 142.07 \\ \hline
        ~ & ~ & ~ \\ \hline
        ~ & ~ & ~ \\ \hline
        Ordnungszahl & Oberschwingungsamplitude [\%] & Oberschwingungsamplitude [V] \\ \hline
        3 & 31.67 & 1119.76 \\ \hline
        9 & 36.02 & 1273.60 \\ \hline
        15 & 15.15 & 535.67 \\ \hline
        21 & 13.43 & 474.85 \\ \hline
        27 & 8.80 & 311.17 \\ \hline
        33 & 9.02 & 318.75 \\ \hline
        39 & 5.40 & 190.97 \\ \hline
        45 & 5.30 & 187.27 \\ \hline
    \end{tabular}
\end{table}
\subsubsection{Oberschwingungsamplituden der Bahnnetzseite}
Die an den umrichterseitigen Wicklungen des Bahnnetztrafos anliegenden Spannungsoberschwingungen sind im nächsten Bild dargestellt. Weiterhin sind die Spannungsoberschwingungen der mittleren Spannung der beiden umrichterseitigen Wicklungen dargestellt. Die Spannungen an den beiden Wicklungen sind zueinander so versetzt, dass sich viele Oberschwingungen in der Summenspannung reduzieren oder auslöschen. Dieser Mittelwert ist die wirksame Gesamtspannung. 

\begin{figure}[htb]
    \centering
    \includegraphics[width=\textwidth]{Bilder/oberschwingungsamp_umrichterseite.png}     
    \caption{Oberschwingungsspektrum der Umrichterspannung der Bahnnetzseite an einer umrichterseitigen Wicklung und mittlere Gesamtspannung der beiden Wicklung}
\end{figure}
\clearpage
\subsection{CAD}
\includepdf[pages=-,landscape]{Bilder/cad.pdf}



% Eigenständigkeitserklärung
%\input{chapter/Eigenstaendigkeitserklaerung}
\end{document}