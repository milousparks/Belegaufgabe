%Dokumentenklasse "scrbook" - Erweitert um den Verweis auf die Verzeichnisse und Texteigenschaften
\documentclass[12pt,a4paper,oneside,parskip=half,listof=totoc,bibliography=totoc,numbers=noendperiod,ngerman,dvipsnames]{scrartcl}
%Anpassung der Seitenränder (Standard bottom ca. 52mm anbzüglich von ca. 4mm für die nach oben rechts gewanderte Seitenzahl)
\usepackage[bottom=48mm,left=25mm,right=25mm]{geometry}

%Tweaks für scrbook
\usepackage{scrhack}

%Blindtext
%\usepackage{blindtext}

%Erlaubt unteranderem Umbrücke captions
\usepackage{caption}
\usepackage{derivative}
%Stichwortverzeichnis
\usepackage{imakeidx}
\usepackage{amsmath}
%Kompakte Listen
\usepackage{paralist}
% Multiline comments
\usepackage{comment}
%Zitate besser formatieren und darstellen
\usepackage{epigraph}
\usepackage{acronym}

%Glossar, Stichworverzeichnis (Akronyme werden als eigene Liste aufgeführt)
\usepackage[toc, acronym]{glossaries} 

%Anpassung von Kopf- und Fußzeile
%beinflusst die erste Seite des Kapitels
\usepackage[automark,headsepline]{scrlayer-scrpage}
%\automark{chapter}
\ihead{\leftmark}
\chead{}
\ohead{\thepage}
\ifoot*{}
\cfoot[\thepage]{}
\cfoot*{}
\ofoot*{\thepage}
\ofoot{Test}
\pagestyle{scrheadings}


%Auskommentieren für die Verkleinerung des vertikalen Abstandes eines neuen Kapitels
%\renewcommand*{\chapterheadstartvskip}{\vspace*{.25\baselineskip}}

%Zeilenabstand 1,5
\usepackage[onehalfspacing]{setspace}

%Verbesserte Darstellung der Buchstaben zueinander
\usepackage[stretch=10]{microtype}

%Deutsche Bezeichnungen für angezeigte Namen (z.B. Innhaltsverzeichnis etc.)
\usepackage[ngerman]{babel}

%Unterstützung von Umlauten und anderen Sonderzeichen (UTF-8)
\usepackage{lmodern}
\usepackage[utf8]{luainputenc}
\usepackage[T1]{fontenc}

%Einfachere Zitate
\usepackage{epigraph}

%Unterstützung der H positionierung (keine automatische Verschiebung eingefügter Elemente)
\usepackage{float} 

%Erlaubt Umbrüche innerhalb von Tabellen
\usepackage{tabularx}

%Erlaubt Seitenumbrüche innerhalb von Tabellen
\usepackage{longtable}

%Erlaubt die Darstellung von Sourcecode mit Highlighting
\usepackage{listings}

%Definierung eigener Farben bei nutzung eines selbst vergebene Namens
\usepackage[table,xcdraw]{xcolor}
\usepackage{pdfpages}
%Vektorgrafiken
\usepackage{tikz}

%Grafiken (wie jpg, png, etc.)
\usepackage{graphicx}

%Grafiken von Text umlaufen lassen
\usepackage{wrapfig}
\usepackage{pgfplots}

%Ermöglicht Verknüpfungen innerhalb des Dokumentes (e.g. for PDF), Links werden durch "hidelink" nicht explizit hervorgehoben
\usepackage[hidelinks,german]{hyperref}
\usepackage{cleveref}
\usepackage{siunitx}
\usepackage{physics}
\usepackage[most]{tcolorbox}
\newtcolorbox{mybox}{
	enhanced,
	boxrule=0pt,frame hidden,
	borderline west={4pt}{0pt}{green!75!black},
	colback=green!10!white,
	sharp corners
}
\newtcolorbox{qed}{
	enhanced,
	boxrule=0pt,frame hidden,
	borderline west={4pt}{0pt}{blue!75!black},
	colback=cyan!5!white,
	sharp corners
}
\usepackage{todonotes}
\usepackage{cancel}%Durchstreichen von sachen
\usepackage[numbered,framed]{matlab-prettifier}
%Einbindung und Verwaltung von Literaturverzeichnissen
\usepackage{csquotes} %wird von biber benötigt
\usepackage[style=ieee, backend=biber, bibencoding=ascii]{biblatex}
\addbibresource{references/references.bib}


\newenvironment{conditions}[1][mit:]
{#1 \begin{tabular}[t]{>{$}l<{$} @{${}={}$} l}}
	{\end{tabular}\\[\belowdisplayskip]}

\usepackage{fancyvrb}
\lstset{literate=% Allow for German characters in lstlistings.
	{Ö}{{\"O}}1
	{Ä}{{\"A}}1
	{Ü}{{\"U}}1
	{ß}{{\ss}}2
	{ü}{{\"u}}1
	{ä}{{\"a}}1
	{ö}{{\"o}}1
}
\usepackage{booktabs}
\usepackage{caption}
\usepackage{subcaption}