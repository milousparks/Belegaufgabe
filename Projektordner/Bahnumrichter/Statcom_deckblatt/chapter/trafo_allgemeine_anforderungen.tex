\section{Anforderungen an 50Hz und 16,7 Hz-Transformatoren}

\subsection{Allgemeine technische Anforderungen}
\begin{table}[htb]
    \centering
    \begin{NiceTabular}{|p{8cm}|p{7cm}|}[hvlines]
        \CodeBefore
        \columncolor{lightergray}{1}
        \Body
        \hline
         Aufstellung & Freiluftaufstellung\\
         \hline
         Verschmutzung & Verschmutzungsgrad III (stark) \\
         \hline
         Aufstellungshöhe & < 1000 m üNN\\
         \hline
         Umgebungstemperatur &  -30°C bis 40°C\\
         \hline
         Klimabedingungen & Normal gem. VDE 0532(IEC76-1)für Freiluftaufstellung\\ 
         \hline
                 \Block{3-1}{Dokumentationen} &  \tabitem Technische Zeichnungen und CAD\\
                         &\tabitem Montageplan, Wartungsplan, Dokumentationen\\
                         &\tabitem Prüfprotokoll der zu erfüllenden Prüfungen\\
                         \hline
                         maximale Kühlmitteltemperatur &  \ang{40}C\\
                         \hline
                         mittl. Wicklungsübertemperatur & \SI{65}{\kelvin} bei Nennleistung\\
                         \hline
                         Übertemperatur Öl oben & \SI{60}{\kelvin}\\
            \hline
            Schalldruckpegel Mittelwert; in 1m Abstand; nach DIN 60551/45635 Teil 30; Nachweis nach VDE0532Teil 7;IEC 551 mit Messung bei Leerlauf.&55dB(A) Tol.\SI[]{0}[]{\percent}\\
            Schalldruckpegel Maximalwert (Einzelwert) ; in 1 m Abstand nach DIN 60551/45635 Teil 30 Nachweis nach VDE0532Teil 7;IEC 551 mit Messung bei Leerlauf. & 60dB(A)Tol.\SI[]{0}[]{\percent}\\ 
                Betriebsart&Dauerbetrieb ;Stromrichterbetrieb\\
                mittl. Wicklungsübertemperatur (Erfahrungswert, durch Hersteller festzulegen)&\SI[]{65}[]{\kelvin}\\
                Übertemperatur Öl oben (Erfahrungswert, durch Hersteller festzulegen)& \SI[]{65}[]{\kelvin}\\
    \end{NiceTabular}
\end{table}

\subsubsection*{Kühlung}

Kühlung: ONAN; 
ONAF vorbereitet (an den Kühlern wird eine Halterung vorgesehen, an der geeignete Lüfter installiert werden können; keine Leistungssteigerung)
\begin{table}[htb]
    \centering
    \begin{NiceTabular}{|l|c|p{5cm}|}
        \CodeBefore
        \columncolor{lightergray}{1}
        \Body
        \hline
        Kühlungsvariante &Innerer Kühlkreislauf &Äußerer Kühlkreislauf \\
        \hline
            ONAN& natürliche Konvektion Öl (Oil Natural)& natürliche Konvektion Umgebungsluft und Wärmestrahlung der Oberfläche (Air Natural)\\
            \hline
            ONAF& natürliche Konvektion Öl (Oil Natural)& erzwungene Konvektion Umgebungsluft und Wärmestrahlung der Oberfläche (Air Forced) \\
            \hline
    \end{NiceTabular}
\end{table}
\subsubsection*{Blechqualität}
durch Hersteller festzulegen
\subsubsection*{Induktion}
bei Nennleerlaufspannung:	$B=\SI[]{1.6}[]{\tesla}$\\
Sättigungskennlinie ist zu liefern

\subsubsection*{Halterungen für eine Traverse für die Kabel zum Umrichter}
Am Transformatorkessel soll  eine Traverse montiert werden, auf der die Kabel zum Umrichter verlegt werden.

\subsubsection*{Angaben zur Betriebsweisen}
\begin{itemize}[noitemsep]
    \item Beide Transformatoren werden im gesamten Bereich $cos(\phi)=0.8\ldots-0.8$ betrieben. Die Transformatoren können sowohl mit gleichförmiger als auch mit stark wechselnder Last beaufschlagt werden (z.B. häufiges Anfahren und Bremsen von Schienenfahrzeugen).
    \item  Die Transformatoren sind über Vakuum Schnellschalter mit dem 110kV 50 Hz bzw. 16,7 Hz Netz verbunden. Die Verbindung kann dabei sowohl über eine Freileitung als auch ein Kabel erfolgen. 
    Im Normalfall werden die Transformatoren nur lastlos abgeschaltet. Schutzabschaltungen können unter Volllast vorkommen und dürfen nicht zu einer Beschädigung der Wicklungen führen. 
    \item Transformator kann über Schnellschalter mit den angeschlossenen Lasten verbunden werden. 
    \item  Betriebsweise 16,7Hz 110kV Netz: Das Netz wird mit Resonanter Sternpunkterdung betrieben. 
    \item  Kurzschlüsse in den angeschlossenen Netzen können betrieblich vorkommen. 

\end{itemize}

\subsubsection*{Kessel}
\begin{itemize}[noitemsep]
    \item Vakuumfeste und öldichte Ausführung (mit Nachweis)
    \item Wandstärke Kessel : bitte angeben	Erfahrungswert : mind. \SI[]{8}[]{\mm}
    \item Wandstärke Boden und Deckel : bitte angeben	Erfahrungswert: mind. \SI[]{10}[]{\mm}
    \item Ausführung als Kasten mit Deckel
    \item Umsetzbares Fahrgestell mit 4 Einfachrollen (einzeln umsetzbar)
    \item Die Lage des Ausdehnungsgefäßes ist gem. den örtlichen Gegebenheiten abzustimmen.
\end{itemize}
\subsubsection*{Kesselzubehör}
\begin{itemize}[noitemsep]
    \item Ölausdehnungsgefäß mit Luftentfeuchter, Ölstandsanzeiger, Absperr- und Entleerungshahn
    \item Anschlussschieber für eine Ölreinigungsanlage (NW 80)
    \item Ölablasshähne für Probeentnahme (oben, mittig, unten)
    \item Ölablassschieber NW 65
    \item Restölablass
    \item Ansatzstellen für hydraulische Hebeböcke in 420 mm Höhe der Ladefläche
    \item Zugösen für alle 4 Fahrtrichtungen
    \item Kranösen
    \item Erdungsschrauben
    \item Sicherheitseinrichtung für das Arbeiten auf dem Deckel
\end{itemize}
\subsubsection*{Ausführung Isolation zwischen Spurkranzrollen  und Kessel}
Die Spurkranzrollen sind zum Kessel hin isoliert anzubringen.
 Bei einer Messspannung von \SI[]{1}[]{\kV} muss der Widerstandswert mindestens \SI[]{10}[]{\mega\ohm} betragen.
 Die Isolierung ist über die gesamte Lebensdauer des Transformators zu gewährleisten. 
\subsubsection*{Kabeltrassen}
Sofern vorhanden sollen am Transformatorkessel Auflagerpunkte für die Kabeltrassen zum Umrichter vorgesehen werden. Auflagerpunke sind Lieferumfang, Kabeltrassen sind nicht Lieferumfang

\subsubsection*{Fahrgestelle}
Die Bodenfreiheit tragender Teile muss mind. 50 mm über SO liegen. Für die Spurkranzrollen wird eine Feststellvorrichtung bei Aufstellung auf dem Fundament vorgesehen. Korrosionsschutz nach DIN 55928 T 1 - 8.

\subsubsection*{Verzinkung}
Siehe Anlage, Spezifikation der DB AG
\subsubsection*{Beschichtungsaufbau}
Siehe Anlage, Spezifikation der DB AG
\subsubsection*{Überwachungseinrichtungen}
\begin{itemize}[noitemsep]
    \item Buchholz-Zweischwimmerrelais für den Kessel mit Warnung und Auslösung 2x(Ö+S)
    \item Luftentfeuchter für den Kessel
    \item Wicklungstempperaturanzeige mit Meßumformer (4...20mA)  mit Warnung und Auslösung 2x(Ö+S)
    \item Ölstandsmelder mit Min./Max.-Kontakt Ö+S
    \item Öltemperaturanzeige  (4...20mA)  und mit Auslösekontakt Ö+S
    \item Thermometertaschen an verschiedenen Stellen des Deckels
    \item Druckentlastungsventile mit Meldekontakt Ö+S
    \item Stromrelais für Kesselschutz und Meldekontakte 2x(Ö+S)
\end{itemize}

\subsubsection*{Kesselschutz / Erdung}
Der Erdungsanschlusspunkt und der Kesselschutzwandler soll in unmittelbarer Nähe zum Klemmkasten angeordnet sein. Anlagenseitig wird das Erdungskabel durch den am Trafo montierten Wandler geführt und am Erdungspunkt angeschlossen. 

\subsubsection*{Anschlusskasten }
Die Verdrahtung der Schutz- und Überwachungsgeräte des Transformators muss geschützt verlegt und in einem spritzwassergeschützten Anschlusskasten IP54 eingeführt werden. Die Gerätebestückung sowie die Anschlussklemmleiste soll den Vorschriften der Deutschen Bahn entsprechen. Die Lage des Kastens ist abzustimmen. Isolierer Aufbau gegenüber dem Kessel.
Ausführungsbeispiel Klemmkasten der DB siehe Anhang.
\subsubsection*{Isolieröl}
Alterungsbeständiges Neuöl, mindestens entsprechend VDE 0370. Das verwendete Öl darf beim Alterungstest nach Baader, DIN 51 554, abgewandelt (110°C, Luft, Cu, 28d) folgende Grenzwerte nicht überschreiten:
Das Trafoöl muss PCB und Chlorfrei sein. Das zugehörige EG-Sicherheitsdatenblatt ist beizulegen.

Typ:	 SHELL-Diala-DX (bevorzugt) oder Nytro Lyra X 

\subsubsection*{Lebensdauer}
Die deutsche Bahn erwartet für alle Komponenten der Umrichteranlage eine Lebensdauer von >20 Jahren. Angaben zur lastabhängigen Lebensdauer des Transformators sind zu machen. 

\subsubsection*{Schallschutz}
Zusatzmaßnahmen zum Schallschutz (z.B. Gummieinlagen, Schenkel lackieren) sind einzubringen

\subsubsection*{Normen}
Ausführung und Prüfung des Transformators nach DIN VDE 0532 /IEC76 
\subsection{Funktionsprüfungen}
\subsubsection*{Prüfungen gemäß DIN VDE 0532- 76-1/ IEC 76}
Prüfungen sind bei Umgebungstemperaturen zwischen 10 °C und 40 °C durchzuführen sowie bei einer
Temperatur des Kühlwassers (falls erforderlich), die $\leq \ang{25}C $.

Die Prüfungen sind im Herstellerwerk vorzunehmen, falls nicht anders zwischen Hersteller und Abnehmer
vereinbart.

Alle äußeren Bestand- und Zubehörteile, die das Verhalten des Transformators bei der Prüfung beeinflussen
können, sind anzubauen.

Für alle Eigenschaften außer der Isolation bilden die Bemessungswerte die Grundlage, wenn nicht die
betreffende Prüfvorschrift anderes festlegt.

Sämtliche bei den Prüfungen verwendeten Messeinrichtungen müssen eine bescheinigte, nachprüfbare
Genauigkeit haben und einer regelmäßigen Kalibrierung unterliegen, entsprechend den Regeln nach
ISO 9001, 4.11.\cite*{DINEN600761.}
\subsubsection*{Stückprüfungen}
\begin{itemize}[noitemsep]
    \item Messung des Wicklungswiderstands
    \item Messung der Übersetzung und Nachweis der Phasendrehung
    \item Messung der Kurzschlussimpedanz und der Kurzschlussverluste 
    \item Messung der Leerlaufverluste und des Leerlaufstroms
    \item Spannungsprüfungen (IEC 60076-3)
    \begin{itemize}
        \item Blitzstoßspannung
        \item Angelegte Stehwechselspannung (ACSD)
        \item Kurzeit-Wechselspannung (ACSD)
    \end{itemize} 
    \item Prüfungen an Stufenschaltern, falls vorhanden
\end{itemize}

\subsubsection*{Typprüfungen}
\begin{itemize}[noitemsep]
    \item Erwärmungsprüfung (IEC 60076-2)
\end{itemize}
\subsubsection*{Sonderprüfungen}
\begin{itemize}[noitemsep]
    \item Spannungsprüfungen (IEC 60076-3)
     \begin{itemize}
        \item Langzeit-Wechselspannung (ACLD)
    \end{itemize}
    \item Bestimmung der Kapazitäten der Wicklungen gegen Erde und zwischen den Wicklungen
    \item Bestimmung des Übertragungsverhaltens von transienten Spannungen
    \item Messung der Nullimpedanz(en) von Drehstromtransformatoren
    \item Nachweis der Kurzschlussfestigkeit
    \item Bestimmung der Geräuschpegel
    \item Messung der Oberschwingungen des Leerlaufstroms
    \item Messung des Isolationswiderstands der Wicklungen gegen Erde und/oder Messung des Verlustfaktors
    ($\tan\theta$) der Kapazitäten des Isoliersystems
\end{itemize}
\subsubsection*{Dokumentation}
Sprache: deutsch\\
zweifache Ausführung in Ordnern\\
zweifache Ausführung elektronisch (CD-ROM)\\

Umfang:
\begin{itemize}[noitemsep]
    \item technische Daten
    \item Prüfnachweise
    \item Inbetriebnahmeanweisung
    \item Wartungsanweisung
    \item Montageanleitung 
    \item Beschreibung inclusive Zeichnungen und Stromlaufplan
\end{itemize}
